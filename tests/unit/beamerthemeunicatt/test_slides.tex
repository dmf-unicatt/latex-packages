\documentclass{beamerunicatt}

\title{My presentation}
\author[F. Ballarin]{\underline{\textbf{Francesco Ballarin}}$^{1,2}$}
\institute[Università Cattolica del Sacro Cuore]{%
$ ^{1}$ Università~Cattolica~del~Sacro~Cuore,\\
\hspace{0.7em} Department~of~Mathematics~and~Physics, Brescia,~Italy\\[1ex]
$ ^{2}$ Università~Cattolica~del~Sacro~Cuore,\\
\hspace{0.7em} Department~of~Mathematics~for~Economic,~Financial\\
\hspace{0.7em} and~Actuarial~Sciences, Milano,~Italy
}
\date{Conference, Location, Date}

\begin{document}

\begin{frame}
\frametitle{Introduction}
\begin{itemize}
\item \highlight{first} item:
\begin{itemize}
\item \highlightB{one} subitem;
\item \highlightB{another} subitem;
\end{itemize}
\item \highlight{second} \highlightB{standalone} item.
\end{itemize}
\end{frame}

\begin{frame}
\frametitle{Some blocks (I)}
\begin{block}{A blue block}
This is a block colored in blue.

\begin{itemize}
\item \highlight{first} item:
\begin{itemize}
\item \highlightB{one} subitem;
\item \highlightB{another} subitem;
\end{itemize}
\item \highlight{second} standalone item.
\end{itemize}
\end{block}

\begin{alertblock}{A red block}
This is a block colored in red.

\begin{itemize}
\item \highlight{first} item:
\begin{enumerate}
\item \highlightB{one} subitem;
\item \highlightB{another} subitem;
\end{enumerate}
\item \highlight{second} standalone item.
\end{itemize}
\end{alertblock}
\end{frame}

\begin{frame}
\frametitle{Some blocks (II)}
\begin{exampleblock}{A green block}
This is a block colored in green.

\begin{enumerate}
\item \highlight{first} item:
\begin{enumerate}
\item \highlightB{one} subitem;
\item \highlightB{another} subitem;
\end{enumerate}
\item \highlight{second} standalone item.
\end{enumerate}
\end{exampleblock}

\begin{remarkblock}{A yellow block}
This is a block colored in yellow.

\begin{enumerate}
\item \highlight{first} item:
\begin{itemize}
\item \highlightB{one} subitem;
\item \highlightB{another} subitem;
\end{itemize}
\item \highlight{second} standalone item.
\end{enumerate}
\end{remarkblock}
\end{frame}

\begin{frame}
\frametitle{Some blocks (III)}
\begin{proofblock}{A gray block}
This is a block colored in gray.

\begin{enumerate}
\item \highlight{first} item:
\begin{itemize}
\item \highlightB{one} subitem;
\item \highlightB{another} subitem;
\end{itemize}
\item \highlight{second} standalone item.
\end{enumerate}
\end{proofblock}
\end{frame}

\begin{frame}
\frametitle{Some results (I)}
\begin{theorem}[Pythagoras]
For a right triangle with sides $a, b$ and hypotenuse $c$, we have
\[
a^2 + b^2 = c^2.
\]
\end{theorem}

\begin{definition}[Prime Number]
A prime number is an integer greater than 1 that has no positive divisors other than 1 and itself.
\end{definition}

\begin{example}[Small Primes]
The first few prime numbers are $2, 3, 5, 7, 11$.
\end{example}

\begin{lemma}[Divisibility Property]
If a prime $p$ divides the product $ab$, then $p$ divides $a$ or $p$ divides $b$.
\end{lemma}
\end{frame}

\begin{frame}
\frametitle{Some results (II)}
\begin{proposition}[Infinitude of Primes]
There are infinitely many prime numbers.
\end{proposition}

\begin{proof}
Assume there are finitely many primes \(p_1,\dots,p_n\). Consider
\[
P = p_1 p_2 \cdots p_n + 1.
\]
Then \(P\) is either prime itself or divisible by a prime not in the list. In either case, there exists a prime not in \(\{p_1,\dots,p_n\}\), contradicting finiteness. Hence, there are infinitely many primes.
\end{proof}

\begin{remark}[Euclid]
The proof of the infinitude of primes, originally due to Euclid, is one of the oldest and most elegant arguments in mathematics.
\end{remark}

\begin{example}[Non-Prime Numbers]
The integers $4, 6, 8, 9, 10$ are not prime, since each can be expressed as a product of smaller integers.
\end{example}
\end{frame}

\section{My first section}

\begin{frame}
\frametitle{Section 1 content}
\begin{block}{Content}
\begin{itemize}
\item This is the \highlight{first} item in the \highlightB{block}.
\item This is the \highlight{second} item in the \highlightB{block}.
\end{itemize}
\end{block}
\end{frame}

\begin{inverseframe}
\frametitle{Inverse frames}

\begin{itemize}
\item \highlight{first} item:
\begin{itemize}
\normalsize
\item \highlightB{one} subitem;
\item \highlightB{another} subitem;
\end{itemize}
\item \highlight{second} \highlightB{standalone} item.
\end{itemize}

\begin{block}{Content}
\begin{itemize}
\item This is the \highlight{first} item in the \highlightB{block}.
\item This is the \highlight{second} item in the \highlightB{block}.
\end{itemize}
\end{block}
\end{inverseframe}

\end{document}
