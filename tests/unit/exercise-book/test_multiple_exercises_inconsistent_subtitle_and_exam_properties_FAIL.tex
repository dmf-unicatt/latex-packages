\documentclass{book}
\usepackage{tests-exercise-book}

\begin{document}

\chapter{Exercises}

\section{All possible combinations of subtitle, examdate, examproblemnumber, examproblemid}

The following exercise is valid: no properties at all.

\begin{exercise}
  What is \(1 + 1\)?
\end{exercise}
\begin{solution}
  \(1 + 1 = 2\)
\end{solution}
\begin{additionalinformation}
Open python and type 1 + 1
\end{additionalinformation}

The following exercise is invalid: only examproblemid provided (must have all three exam properties together).

\begin{exercise}[examproblemid={2020-01-15-02}]
  What is \(2 + 2\)?
\end{exercise}
\begin{solution}
  \(2 + 2 = 4\)
\end{solution}
\begin{additionalinformation}
Open python and type 2 + 2
\end{additionalinformation}

The following exercise is invalid: only examproblemnumber provided (must have all three exam properties together).

\begin{exercise}[examproblemnumber={3}]
  What is \(3 + 3\)?
\end{exercise}
\begin{solution}
  \(3 + 3 = 6\)
\end{solution}
\begin{additionalinformation}
Open python and type 3 + 3
\end{additionalinformation}

The following exercise is invalid: examproblemnumber and examproblemid provided without examdate.

\begin{exercise}[examproblemnumber={4}, examproblemid={2020-01-15-04}]
  What is \(4 + 4\)?
\end{exercise}
\begin{solution}
  \(4 + 4 = 8\)
\end{solution}
\begin{additionalinformation}
Open python and type 4 + 4
\end{additionalinformation}

The following exercise is invalid: only examdate provided (must have all three exam properties together).

\begin{exercise}[examdate={January 15, 2020}]
  What is \(5 + 5\)?
\end{exercise}
\begin{solution}
  \(5 + 5 = 10\)
\end{solution}
\begin{additionalinformation}
Open python and type 5 + 5
\end{additionalinformation}

The following exercise is invalid: examdate and examproblemid provided without examproblemnumber.

\begin{exercise}[examdate={January 15, 2020}, examproblemid={2020-01-15-06}]
  What is \(6 + 6\)?
\end{exercise}
\begin{solution}
  \(6 + 6 = 12\)
\end{solution}
\begin{additionalinformation}
Open python and type 6 + 6
\end{additionalinformation}

The following exercise is invalid: examdate and examproblemnumber provided without examproblemid.

\begin{exercise}[examdate={January 15, 2020}, examproblemnumber={7}]
  What is \(7 + 7\)?
\end{exercise}
\begin{solution}
  \(7 + 7 = 14\)
\end{solution}
\begin{additionalinformation}
Open python and type 7 + 7
\end{additionalinformation}

The following exercise is valid: all three exam properties provided.

\begin{exercise}[examdate={January 15, 2020}, examproblemnumber={8}, examproblemid={2020-01-15-08}]
  What is \(8 + 8\)?
\end{exercise}
\begin{solution}
  \(8 + 8 = 16\)
\end{solution}
\begin{additionalinformation}
Open python and type 8 + 8
\end{additionalinformation}

The following exercise is valid: only subtitle provided.

\begin{exercise}[subtitle={Simple Addition}]
  What is \(9 + 9\)?
\end{exercise}
\begin{solution}
  \(9 + 9 = 18\)
\end{solution}
\begin{additionalinformation}
Open python and type 9 + 9
\end{additionalinformation}

The following exercise is invalid: subtitle and examproblemid provided.

\begin{exercise}[subtitle={Simple Addition}, examproblemid={2020-01-15-10}]
  What is \(10 + 10\)?
\end{exercise}
\begin{solution}
  \(10 + 10 = 20\)
\end{solution}
\begin{additionalinformation}
Open python and type 10 + 10
\end{additionalinformation}

The following exercise is invalid: subtitle and examproblemnumber provided.

\begin{exercise}[subtitle={Simple Addition}, examproblemnumber={11}]
  What is \(11 + 11\)?
\end{exercise}
\begin{solution}
  \(11 + 11 = 22\)
\end{solution}
\begin{additionalinformation}
Open python and type 11 + 11
\end{additionalinformation}

The following exercise is invalid: subtitle and examdate provided.

\begin{exercise}[subtitle={Simple Addition}, examdate={January 15, 2020}]
  What is \(12 + 12\)?
\end{exercise}
\begin{solution}
  \(12 + 12 = 24\)
\end{solution}
\begin{additionalinformation}
Open python and type 12 + 12
\end{additionalinformation}

The following exercise is invalid: subtitle, examproblemnumber, and examproblemid provided.

\begin{exercise}[subtitle={Simple Addition}, examproblemnumber={13}, examproblemid={2020-01-15-13}]
  What is \(13 + 13\)?
\end{exercise}
\begin{solution}
  \(13 + 13 = 26\)
\end{solution}
\begin{additionalinformation}
Open python and type 13 + 13
\end{additionalinformation}

The following exercise is invalid: subtitle, examdate, and examproblemid provided.

\begin{exercise}[subtitle={Simple Addition}, examdate={January 15, 2020}, examproblemid={2020-01-15-14}]
  What is \(14 + 14\)?
\end{exercise}
\begin{solution}
  \(14 + 14 = 28\)
\end{solution}
\begin{additionalinformation}
Open python and type 14 + 14
\end{additionalinformation}

The following exercise is invalid: subtitle, examdate, and examproblemnumber provided.

\begin{exercise}[subtitle={Simple Addition}, examdate={January 15, 2020}, examproblemnumber={15}]
  What is \(15 + 15\)?
\end{exercise}
\begin{solution}
  \(15 + 15 = 30\)
\end{solution}
\begin{additionalinformation}
Open python and type 15 + 15
\end{additionalinformation}

The following exercise is invalid: subtitle, examdate, examproblemnumber, and examproblemid all provided.

\begin{exercise}[subtitle={Simple Addition}, examdate={January 15, 2020}, examproblemnumber={16}, examproblemid={2020-01-15-16}]
  What is \(16 + 16\)?
\end{exercise}
\begin{solution}
  \(16 + 16 = 32\)
\end{solution}
\begin{additionalinformation}
Open python and type 16 + 16
\end{additionalinformation}

\chapter{Solutions}
\AssertPropertyInAllExercises{solution}
\PrintExercisePropertySectionBySection{solution}

\chapter{Additional information}
\AssertPropertyInAllExercises{additionalinformation}
\PrintExercisePropertySectionBySection{additionalinformation}

\end{document}
