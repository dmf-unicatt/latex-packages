\documentclass{book}
\usepackage{tests-exercise-book}

\begin{document}

\chapter{Exercises}

\section{First section}

\subsection{First subsection}

\begin{exercise}[subtitle={Simple Addition}]
  What is \(2 + 2\)?

  Please answer.
\end{exercise}

\begin{solution}
Using LaTex I know that

  \(2 + 2 = 4\)
\end{solution}

\begin{additionalinformation}
Open python and do the following:

\begin{pycell}
2 + 2
\end{pycell}
\begin{pyexpectedoutput}
4
\end{pyexpectedoutput}
\end{additionalinformation}

\section{Second section}

\subsection{A subsection}

\begin{exercise}
  What is \(3 + 3\)?
\end{exercise}

\begin{solution}
  \(3 + 3 = 6\)

Want to try with python?
\begin{pycell}
3 + 3
\end{pycell}
\begin{pyexpectedoutput}
6
\end{pyexpectedoutput}
\end{solution}

\begin{additionalinformation}
You can proceed as before, say
\begin{pycell}
3 + 3
\end{pycell}
\begin{pyexpectedoutput}
6
\end{pyexpectedoutput}

Still, there is also a more elaborated solution with variables:

\begin{pycell}
a = 3
b = 3
a + b
\end{pycell}
\begin{pyexpectedoutput}
6
\end{pyexpectedoutput}
\end{additionalinformation}

\subsection{Another subsection}

\begin{exercise}
  What is \(13 + 3\)?
\end{exercise}

\begin{solution}
  \(13 + 3 = 16\)
\end{solution}

\begin{additionalinformation}
A more elaborated solution with functions:

\begin{pycell}
def my_sum(a, b):
    return a + b

print(my_sum(13, 3))
\end{pycell}
\begin{pyexpectedoutput}
16
\end{pyexpectedoutput}
\end{additionalinformation}

\begin{exercise}
Discuss this code

\begin{pycell}
13 + 3
\end{pycell}
\begin{pyexpectedoutput}
16
\end{pyexpectedoutput}
\end{exercise}

\begin{solution}
It's the sum of two numbers, and prints 16.
\end{solution}

\begin{additionalinformation}
A more elaborated solution with a plot. Store the numbers in a list.

\begin{pycell}
numbers = [3, 13]
\end{pycell}

Then plot their cumulative sum.

\begin{pycell}
import plotly.express as px
fig = px.scatter(
    x=[1, 2],
    y=[numbers[0], numbers[0] + numbers[1]]
)
fig
\end{pycell}
\pyexpectedfigure{images/scatter_12_316.png}

\newpage

You can plot again the figure if you want.

\begin{pycell}
print(sum(numbers))
fig
\end{pycell}
\begin{pyexpectedoutput}
16
\end{pyexpectedoutput}
\pyexpectedfigure{images/scatter_12_316.png}
\end{additionalinformation}

\chapter{Solutions}
\AssertPropertyInAllExercises{solution}
\PrintExercisePropertySectionBySection{solution}

\chapter{Additional information}
\AssertPropertyInAllExercises{additionalinformation}
\PrintExercisePropertySectionBySection{additionalinformation}

\end{document}
