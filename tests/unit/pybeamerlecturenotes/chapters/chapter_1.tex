\setcounter{chapter}{0}
\chapter{Introduction}
\label{chap:intro}

\begin{frame}
\frametitle{Introduction}
\begin{itemize}
\item \highlight{first} item:
\begin{itemize}
\item \highlightB{one} subitem;
\item \highlightB{another} subitem;
\end{itemize}
\item \highlight{second} \highlightB{standalone} item.
\end{itemize}
\end{frame}

\begin{frame}
\frametitle{Introduction (II)}
\begin{itemize}
\item \highlight{third} item:
\begin{itemize}
\item \highlightB{one} subitem;
\item \highlightB{another} subitem;
\end{itemize}
\item \highlight{fourth} \highlightB{standalone} item.
\end{itemize}
\end{frame}

\begin{frame}
\frametitle{Introduction (III)}
This is a sentence that will appear in both book and slides.

\begin{equation}
1 + 1 = 2
\label{eq:1p1e2_chap1}
\end{equation}
\end{frame}

\begin{frame}
\frametitle{Introduction (IV)}
This is another sentence that will appear in both book and slides.

Furthermore, we can refer to
\begin{itemize}
\item the current chapter: \ref{chap:intro},
\item a chapter in another file: \ref{chap:inverse_frames},
\item a formula in a slide in the current file: \eqref{eq:1p1e2_chap1},
\item a formula in a slide in another file: \eqref{eq:1p1e2_chap2},
\item a lemma in a slide in another file: Lemma \ref{lemma:divisibility_property},
\item a figure in a slide in another file: Figure \ref{fig:pythagorean}.
\end{itemize}
\end{frame}

\mode<article>{
This content will only be shown in the book, and not in the slides
}

