\setcounter{chapter}{2}
\chapter{Theorem-like environments}

\begin{frame}
\frametitle{Some results (I)}
\begin{theorem}[Pythagoras]
For a right triangle with sides $a, b$ and hypotenuse $c$, we have
\[
a^2 + b^2 = c^2.
\]
\end{theorem}

\begin{definition}[Prime Number]
A prime number is an integer greater than 1 that has no positive divisors other than 1 and itself.
\end{definition}

\begin{example}[Small Primes]
The first few prime numbers are $2, 3, 5, 7, 11$.
\end{example}

\begin{lemma}[Divisibility Property]
If a prime $p$ divides the product $ab$, then $p$ divides $a$ or $p$ divides $b$.
\end{lemma}
\end{frame}


\begin{frame}
\frametitle{Some results (II)}
\begin{proposition}[Infinitude of Primes]
There are infinitely many prime numbers.
\end{proposition}

\begin{proof}
Assume there are finitely many primes \(p_1,\dots,p_n\). Consider
\[
P = p_1 p_2 \cdots p_n + 1.
\]
Then \(P\) is either prime itself or divisible by a prime not in the list. In either case, there exists a prime not in \(\{p_1,\dots,p_n\}\), contradicting finiteness. Hence, there are infinitely many primes.
\end{proof}

\begin{remark}[Euclid]
The proof of the infinitude of primes, originally due to Euclid, is one of the oldest and most elegant arguments in mathematics.
\end{remark}

\begin{example}[Non-Prime Numbers]
The integers $4, 6, 8, 9, 10$ are not prime, since each can be expressed as a product of smaller integers.
\end{example}
\end{frame}
