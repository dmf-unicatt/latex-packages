\documentclass{book}
\usepackage{tests-tex-notebook}

\ifPythonTeXLoaded
\restartpythontexsession{\thechapter}
\makeatletter
\@addtoreset{linenum}{chapter}
\makeatother
\else
\usepackage{etoolbox}
\pretocmd{\chapter}{\stepcounter{lastpynotebook}}{}{}
\fi

\begin{document}

\chapter{Case with numbers}

\begin{pycell}
x = 1
y = 2
\end{pycell}

\begin{pycell}
z = 3; n = 4
\end{pycell}

\begin{pycell}
l = 6
m = 7
\end{pycell}


We can print a linear combination of the provided number from a pycell:

\begin{itemize}
\item directly
\begin{pycell}
print(x + y + z + n - l - m)
\end{pycell}
\begin{pyexpectedoutput}
-3
\end{pyexpectedoutput}
\item implicitly
\begin{pycell}
x + y + z + n - l - m
\end{pycell}
\begin{pyexpectedoutput}
-3
\end{pyexpectedoutput}
\item in both ways (with duplicated output)
\begin{pycell}
print(x + y + z + n - l - m + 1)
x + y + z + n - l - m + 2
\end{pycell}
\begin{pyexpectedoutput}
-2
-1
\end{pyexpectedoutput}
\end{itemize}

One can also print lists of numbers
\begin{pycell}
x = [1, 2, 3]
y = [4, 5, 6]
print(f"{x}")
print(f"{y}")
\end{pycell}
\begin{pyexpectedoutput}
[1, 2, 3]
[4, 5, 6]
\end{pyexpectedoutput}


\chapter{Case with strings and spaces}

\begin{pycell}
x = "one"
y = "two two"
\end{pycell}

\begin{pycell}
z = "three THREE_Three"; n = "FoUr_four quattro_cuatro"
\end{pycell}

\begin{pycell}
" ".join([x, y, z, n])
\end{pycell}
\begin{pyexpectedoutput}
'one two two three THREE_Three FoUr_four quattro_cuatro'
\end{pyexpectedoutput}

\begin{pycell}
print(" ".join([x, y]))
print(" ".join([z, n]))
\end{pycell}
\begin{pyexpectedoutput}
one two two
three THREE_Three FoUr_four quattro_cuatro
\end{pyexpectedoutput}

Once can also print multiline strings
\begin{pycell}
Z = """three
THREE_Three"""
N = """FoUr_four
quattro_cuatro"""
print("\n".join([Z, N]))
\end{pycell}
\begin{pyexpectedoutput}
three
THREE_Three
FoUr_four
quattro_cuatro
\end{pyexpectedoutput}

\begin{pycell}
Z = '''three
THREE_Three'''
N = '''FoUr_four
quattro_cuatro'''
print("\n".join([Z, N]))
\end{pycell}
\begin{pyexpectedoutput}
three
THREE_Three
FoUr_four
quattro_cuatro
\end{pyexpectedoutput}

\chapter{Variable resets}

You can use \texttt{\textbackslash restartpythontexsession} to control when the python session resets. For instance, here it resets at every chapter, and therefore

\begin{pycell}
a = 1
b = "this is a string"
print(a, b)
\end{pycell}
\begin{pyexpectedoutput}
1 this is a string
\end{pyexpectedoutput}

works, but trying to use the variable \texttt{x} from the previous chapter would not (see \texttt{test\_pycell\_FAIL.tex}).

\chapter{Figures}

You can have plotly figures in the code cells. By outputting the figure in the last line of the code cell it will be automatically added to the document.

\begin{pycell}
import plotly.express as px
fig = px.scatter(x=[1, 2, 3], y=[4, 5, 6])
fig
\end{pycell}
\pyexpectedfigure{images/scatter_123_456.png}

A figure appears also when other text outputs are present
\begin{pycell}
x = [10, 20, 30]
y = [40, 50, 60]
print(f"x = {x}")
print(f"y = {y}")
fig = px.scatter(x=x, y=y)
fig
\end{pycell}
\begin{pyexpectedoutput}
x = [10, 20, 30]
y = [40, 50, 60]
\end{pyexpectedoutput}
\pyexpectedfigure{images/scatter_102030_405060.png}

\end{document}
